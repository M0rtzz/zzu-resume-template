% !TEX program = XeLaTeX
% !TEX encoding = UTF-8
% @brief: resume template in XeLaTeX
% @author: M0rtzz
% @reference: a template from https://overleaf.com
% @date: 2024-09-17
% @version: 1.0
% @blog: LaTeX配置教程:https://www.m0rtzz.com/posts/4

\documentclass[11pt]{article}

% 全局禁用段落首行缩进
\setlength{\parindent}{0pt}

% 一些扩展功能和优化,定义了XeTeX Logo
\usepackage{xltxtra}
\usepackage{bookmark}

% 使用hyperlink显示email和url
\usepackage{hyperref}
\hypersetup{hidelinks}
\usepackage{url}
\urlstyle{tt}
\usepackage{multicol}
\usepackage{xcolor}

% 统一一种颜色,偏蓝色,用于section下划线和font-awesome
\definecolor{ZZU_BLUE}{RGB}{0, 63,  148} % 从ZZU校徽上取的蓝色

% \widthof[]{} 用于特殊对齐时用到
\usepackage{calc}

% 利用tikz来定位照片和学校Logo
\usepackage{graphicx}
\usepackage{tikz}
\usetikzlibrary{calc}

% 加载字体
\usepackage{fontspec}
\usepackage{xeCJK}

% 取消中文与数字之间间隙
\CJKsetecglue{}

% 字体注意版权问题,这两种字体应该比较好看,英文Helvetica,中文方正兰亭黑,也是有多种版本,自己试试哪些好看。

% windows好像需要先安装字体,之后使用下面语句就够了
% main document font

% \setmainfont[
%     BoldFont = HelveticaNeueLTPro-Md.otf ,
% ]{HelveticaNeueLTPro-Roman.otf}
% \setCJKmainfont[
%     BoldFont=Pro_GB18030-DemiBold.otf,
% ]{Pro_GB18030.otf}

% linux只需要字体路径就行了,如下
% main document font
\setmainfont[
    Path = ./fonts/,
    Extension = .otf,
    BoldFont = HelveticaNeueLTPro-Md.otf,
]{HelveticaNeueLTPro-Roman.otf}
\setCJKmainfont[
    Path = ./fonts/,
    Extension = .otf,
    BoldFont=ProGB18030-DemiBold.otf,
]{ProGB18030.otf}

% 定义更漂亮的“C++”,参考https://tex.stackexchange.com/questions/4302/prettiest-way-to-typeset-c-cplusplus 
% 貌似跟具体字体大小有关,需要调下参数,我测试感觉下面的比较好看
\usepackage{relsize}
\usepackage{xspace}
\protected\def\CC{{C\nolinebreak[4]\hspace{-.05em}\raisebox{.28ex}{\relsize{-1}++}}\xspace}

% use fontawesome-4.7.0
\usepackage{fontawesome-4.7.0}
\makeatletter
\DeclareFontShape{TU}{FontAwesome-4.7.0.otf(0)}{b}{n}{<->ssub*FontAwesome-4.7.0.otf(0)/m/n}{}
\makeatother

\usepackage[
    a4paper,
    left=1.2cm,
    right=1.2cm,
    top=1.5cm,
    bottom=1cm,
    nohead
]{geometry}
\renewcommand{\baselinestretch}{1.1} % 定义行间距1.1,根据需要调整
\usepackage{titlesec}
\usepackage{enumitem}

\setlist{noitemsep} % 去除列表项之间的间距,但保留整个列表周围的间距
% \setlist{nosep} % 删除列表内部及周围的全部垂直间距

% \setlist[itemize]{topsep=0.25em, leftmargin=*} % 对“itemize”环境设置:顶部和底部间距为0.25em,左侧缩进与文本对齐
% \setlist[enumerate]{topsep=0.25em, leftmargin=*} % 对“enumerate”环境设置:顶部和底部间距为0.25em,左侧缩进与文本对齐
\setlist[itemize]{topsep=0.1em, leftmargin=*}
\setlist[enumerate]{topsep=0.2em, leftmargin=*}

\titleformat{\section} % 自定义 \section 命令格式
{\large\bfseries\raggedright} % 设置 \section 标题样式为较大字体(\large)、粗体(\bfseries)并左对齐(\raggedright)
{}{0em} % 这里可以用来为所有小节添加前缀,如 '第...节'
{} % 这里可以插入代码,在标题前面显示内容
[{\color{ZZU_BLUE}\titlerule}] % 在每个小节标题后插入一条颜色为 ZZU_BLUE 的水平线
% \titlespacing*{\section}{0cm}{*1.0}{*1.0}
\titlespacing*{\section}{0cm}{*1.6}{*1.2}
\usepackage{siunitx}
\usepackage{amssymb}

% \xeCJKsetup{CJKspace=true}
% \xeCJKDeclareCharClass{CJK}{`0 -> `9} % 设置 0-9 以 CJK 字体输出
% \normalspacedchars{0,1,2,3,4,5,6,7,8,9} % 0-9 的字符类被还原

\begin{document}
\pagenumbering{gobble} % 隐藏页码显示

% 利用tikz来定位照片,部分招聘单位可能需要“以貌取人”
\begin{tikzpicture}[remember picture, overlay]
    \node[anchor = north east] at ($(current page.north east)+(-1cm,-1.2cm)$) {\includegraphics[height=2.8cm]{./images/one-inch_blue_background.pdf}};
\end{tikzpicture}%

% 利用tikz来定位学校Logo,这里只在第一页显示,如果需要每页都有,可以考虑在页眉、页脚或者background中加入,不过简历也就一两页,无所谓了
\begin{tikzpicture}[remember picture, overlay]
    \node[anchor = north west] at ($(current page.north west)+(0.2cm,-0.2cm)$) {\includegraphics[height=1.5cm]{./images/zzu_new.pdf}};
\end{tikzpicture}

% 利用tikz来定位页脚栏(不喜欢可以直接注释掉),电子版简历使用,黑白纸质打印效果可能并不好。这里只在第一页显示,如果需要每页都有,页脚或者backgroud宏包中加入
\begin{tikzpicture}[remember picture, overlay]
    \node[anchor = south,fill=ZZU_BLUE,draw=none,minimum width=\paperwidth,minimum height=1.5em,align=center,font=\footnotesize,text=white] at ($(current page.south)$)
    {\faGithubAlt \ \href{https://github.com/XX}{https://github.com/XX}\qquad
        \faRssSquare \ \href{https://XX.github.io}{https://XX.github.io} };
\end{tikzpicture}

% tikzpicture环境很敏感,注释周围的空格、空行都会引起水平距离或垂直距离的变化,
\centerline{\LARGE\bfseries{M0rtzz}}
\centerline{\normalsize{20岁 \hspace{0.5ex}|\hspace{0.5ex} 男 \hspace{0.5ex}|\hspace{0.5ex} 中共党员}}

\centerline{\normalsize{
        \faPhone \ 123 4567 8900 \quad
        \faEnvelopeO \ \href{mailto:your@email.com}{your@email.com}}}
% 最好用你的edu邮箱

\centerline{\normalsize{
        \faExternalLink \ \href{https://XX.github.io}{https://XX.github.io} \quad
        \faGithubAlt \ \href{https://github.com/XX}{https://github.com/XX}
    }}
\vspace{1.5ex}

% \section{\makebox[\widthof{\faGraduationCap}][c]{\color{ZZU_BLUE}\faGraduationCap}\  教育背景}
\section{\texorpdfstring{\makebox[\widthof{\faGraduationCap}][c]{\color{ZZU_BLUE}\faGraduationCap}\ 教育经历}{教育经历}}

{\bfseries 郑州大学(211 \textbar{} 双一流)}\enspace 本科 \enspace 计算机与人工智能学院 \enspace 计算机科学与技术专业 \hfill XXXX年XX月\hspace{0.5em}--\hspace{0.5em}至今

% \section{\makebox[\widthof{\faGraduationCap}][c]{\color{ZZU_BLUE}\faTrophy}\ 获奖情况}
\section{\texorpdfstring{\makebox[\widthof{\faTrophy}][c]{\color{ZZU_BLUE}\faTrophy}\ 获奖情况}{获奖情况}}

\begin{itemize}
    \item {专业类:XX比赛 \ \textendash \ XX组{\bfseries 冠军(国家级一等奖)}、XX比赛{\bfseries 冠军(国家级一等奖)}、XX比赛{\bfseries 季军(国家级一等奖)}}
    \item {创新创业:{\bfseries 国家级大学生创新创业训练计划项目结题},郑州大学创新创业基础与工程设计实践项目{\bfseries 队长}\ldots}
    \item {荣誉类:{\bfseries 郑州大学科技创新先进个人代表(全校10人)}、郑州大学一等奖学金、XX奖学金、校级三好学生\ldots}
\end{itemize}

% \section{\makebox[\widthof{\faGraduationCap}][c]{\color{ZZU_BLUE}\faFileText}\ 学习成绩}
\section{\texorpdfstring{\makebox[\widthof{\faFileText}][c]{\color{ZZU_BLUE}\faFileText}\ 学习成绩}{学习成绩}}

{\bfseries 平均绩点:$\mathsf{3.X_{/4}}$} \hspace{0.5em}
{\bfseries 总平均分:$\mathsf{X_{/100}}$} \hspace{0.5em}
{\bfseries 专业排名:$\mathsf{X_{/191}}$ (专业前1\%)}
\begin{itemize}[parsep=0.5ex]
    \item {专业类:微积分-99、计算机网络-96、数据结构-95、计算机组成原理-97、创新创业基础与工程设计实践-99\ldots}
    \item {英语:通过 四级(CET-4)、六级(CET-6)}
\end{itemize}

% \section{\makebox[\widthof{\faGraduationCap}][c]{\color{ZZU_BLUE}\faUniversity}\ 实践经历}
\section{\texorpdfstring{\makebox[\widthof{\faUniversity}][c]{\color{ZZU_BLUE}\faUniversity}\ 研究经历}{研究经历}}

{\bfseries XX机器人研制(郑州大学XX实验室负责人)} \hfill XXXX年XX月\hspace{0.5em}--\hspace{0.5em}至今
\begin{itemize}
    \item {\ldots\ldots\ldots}
    \item {\ldots\ldots\ldots}
    \item {\ldots\ldots\ldots}
    \item {\ldots\ldots\ldots}
    \item {\ldots\ldots\ldots}
\end{itemize}

{\bfseries 面向XX的XX方法研究} \hfill XXXX年XX月\hspace{0.5em}--\hspace{0.5em}至今
\begin{itemize}
    \item {\ldots\ldots\ldots}
    \item {\ldots\ldots\ldots}
    \item {\ldots\ldots\ldots}
    \item {\ldots\ldots\ldots}
    \item {\ldots\ldots\ldots}
\end{itemize}

{\bfseries 国家级大学生创新项目\hspace{0.2em} | \hspace{0.2em}XX}\hfill XXXX年XX月\hspace{0.5em}--\hspace{0.5em}XXXX年XX月
\begin{itemize}
    \item {\ldots\ldots\ldots}
    \item {\ldots\ldots\ldots}
    \item {\ldots\ldots\ldots}
    \item {\ldots\ldots\ldots}
    \item {\ldots\ldots\ldots}
\end{itemize}

% {\bfseries 交流项目}
% \begin{itemize}
%     \item {郑州大学(University of XX)(202X年X月)}
%     \item {YY大学(University of YY)(202X年X月)}
% \end{itemize}

% increase linespacing [parsep=0.5ex]
% \begin{itemize}[parsep=0.5ex]
%     \item {编程语言: C == Python > \CC > Java}
%     \item {平台: Linux}
%     \item {开发: 英语六级,在读期间阅读了大量专业英文文献、开源项目英文文档等。}
% \end{itemize}

% \section{\makebox[\widthof{\faGraduationCap}][c]{\color{ZZU_BLUE}\faUsers}\ 学生工作}
\section{\texorpdfstring{\makebox[\widthof{\faUsers}][c]{\color{ZZU_BLUE}\faUsers}\ 工作经历}{工作经历}}

\begin{itemize}
    \item {\ldots\ldots\ldots}
    \item {\ldots\ldots\ldots}
\end{itemize}

% \section{\makebox[\widthof{\faGraduationCap}][c]{\color{ZZU_BLUE}\faWrench}\ 工作技能}
% \section{\texorpdfstring{\makebox[\widthof{\faGraduationCap}][c]{\color{ZZU_BLUE}\faWrench}\ 工作技能}{工作技能}}

% \begin{itemize}
%     \item {教师资格证:高中数学,合格}
%     \item {普通话证书:二级甲等}
%     \item {文字排版:\LaTeX{}}
%     \item {编程:Matlab,Python,R,Geogebra}

% \end{itemize}

% \section{\makebox[\widthof{\faGraduationCap}][c]{\color{ZZU_BLUE}\faTags}\ 其他}
% \section{\texorpdfstring{\makebox[\widthof{\faTags}][c]{\color{ZZU_BLUE}\faTags}\ 其他}{其他}}
\section{\texorpdfstring{\makebox[\widthof{\faIdCard}][c]{\color{ZZU_BLUE}\faIdCard}\ 自我评价}{自我评价}}
% \section{\texorpdfstring{\makebox[\widthof{\faWrench}][c]{\color{ZZU_BLUE}\faWrench}\ 其他}{其他}}
% increase linespacing [parsep=0.5ex]

\begin{itemize}
    \item {掌握 C/\CC、Python、ROS、PyTorch深度学习框架、Linux OS、Shell、Git、Docker、Nginx、\LaTeX \ 等技术。}
    \item {对科研抱有极高热情,在工作中习惯与合作者定期开会探讨科研问题,抗挫折能力强,心态乐观,尽职尽责。}
    \item {对计算机知识热忱,动手能力强,能独立快速部署环境并复现实验、模型、框架及开源库并独立解决问题。}
\end{itemize}

% 如果多页简历,可以手动在适当位置插入 \newpage 或者 \clearpage 开始新一页

\end{document}
